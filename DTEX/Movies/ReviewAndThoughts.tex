\documentclass[12pt]{extarticle}
\usepackage{amsmath}
\usepackage{tikz}

\usepackage{amsfonts}
\usepackage{amssymb}
\usepackage{graphicx}
\usepackage{mathtools}
\usepackage{color}
\usepackage{hyperref}

\usepackage[margin=0.5in]{geometry}
\newcommand\tab[1][1cm]{\hspace*{#1}}
\usepackage{collectbox}
\DeclarePairedDelimiter\ceil{\lceils}{\rceil}
\DeclarePairedDelimiter\floor{\lfloor}{\rfloor}
\usepackage{amsthm}
%opening
\title{Media}
\author{Daniel Drake}

\theoremstyle{plain}
\newtheorem{thm}{Theorem}[section] 

\theoremstyle{plain}
\newtheorem{axiom}{Axiom}[section] 

\theoremstyle{plain}
\newtheorem{lma}{Lemma}[section] 

\theoremstyle{Definition}
\newtheorem{def.}{Definition}[section] 

\theoremstyle{Definition}
\newtheorem{prf}{Proof}[section] 

\theoremstyle{plain}
\newtheorem{exmp}{Example}[section]

\theoremstyle{plain}
\newtheorem{ruleOfInference}{Rule Of Inference}[section]

\newcommand{\cut}[0]{\noindent\framebox[\linewidth]{\rule{\linewidth}{2pt}}\\}
\newcommand{\prof}[0]{	\noindent \textbf{Proof:} \rule{500pt}{2pt} \\ }

\newcommand{\ddash}{\boxed{\vdash}}

\begin{document}
	\maketitle
	\section{Dead Pool 2}
	Just watched the movie for the first time in a while. Overall it is still an entertaining movie. Deadpool is a great character. Talking shit and murdering in funny ways as always. Domino was a great character. Luck is an awesome super power and it played really well into the film. It worked especially well as it kinda breaks the fourth wall. That wouldn't work well in any other movie that a Deadpool movie. The kid was dumb but I guess they needed some kind of a plot point for the movie to be about so whatever. As the kid being the plot point, the time travel stuff was irrelevant and I really didn't mind to much. The best use of time travel actually happened at the end after the credits. A better plot would have been to have Deadpool after the guy who killed his girlfriend and have Cable being needing to stop Deadpool for some reason. Maybe that would have been cliche though so whatever. The plot was ultimately about revenge anyway so either way, it would have kept that going on. Dopinder was a great character. He is kinda like Deadpool's dog which is great. Kinda makes me think there will be an Indian super hero at some point though. Or maybe they could actually give Dopinder some powers. That would be hilarious. Overall I found myself skipping some scenes. The monologuing when Deadpool was dying is one example. Except the part when he talked to Cable. That was hilarious. Cable was good, but he felt rather under-powered. Further, his comedic points could have been better. Maybe have him do a deadpan dad joke or something out of nowhere. Colossus is a perfect character, I wouldn't change a thing. I probably will watch this again as the comedy will be fresh again as this movie is rather surprising and somewhat forgettable which in this case is a good thing. Didn't really think much about the other characters. They were great in the origional watch but nothing really to write about after watching this movie like 4 times or whatever after it had come out. 
	
	\section{Blow}
	After re-watching blow it was a roller coaster of a ride. It starts out slow rises to the greatest high and then slowly burns out. I'm not sure if this was intentionally done by the directors, but it was definitely there. As such the first 75 percent of the movie is great. The ending is sad and was still sad after re-watching it. The logical progression of the story is pretty obvious, but it seems getting George who is the main character to give out his source after getting shot is something that could have happened, but unlikely. That kind of thing seems like an accident or by force rather than being brought on by some trauma so extreme that it would put most people in shock incapable of higher reasoning. The fun parts are fun, the sad parts are sad and since I knew the ending already, I just skipped it. The sets were all really well done and the acting was great. Not much else to say about this film. 
	
	\section{Serial Experiments Lain}
		\subsection{Layer 01}
			Serial Experiments Lain starts of with a question, "Why won't you come here?"
			Even more striking than that is the artwork. Neon colors, glitching computer graphics and psychedelic textures layered tastefully throughout the show, it is great. The first episode deals with Chisa Yomoda's suicide and the reason behind it. The show seems to get at this person killed herself in order to enter the wired. The sounds in the show are interesting as well, pulsing sin waves and random chattering make for a great background to set a feeling of cosmological technological horror. Lain being told to check her email is the beginning of the journey that lain takes to becoming whatever lain is by the end. So in a way, Chisa died not only to enter the wired which she uses to email lain, but so that Lain can herself start messing with computers.  The first OS she uses is Communication OS version C? It demonstrates simple graphics but does use the cut circles so prevalent in this series. The OS has a voice interface and a very active graphics animation system. So Lain's father specifically talks about the wired as a world existing outside of the real world. He is running the OYJ program which appears to have 3D modeling capabilities as well as networking capabilities. This program is running on the Oyajiization System, the apparent OS on his computer. Lain then goes on to tell him that she has a friend she want to see. This is Lain saying that should would like to see Chisa Yomoda in the wired. Lain spends some time staring at the array of power and communication lines wired everywhere. This is super interesting since given the resent activity, she may be wondering if Chisa is actually a ghost in those wires. And so the episode ends with Chisa fading and with some more networking wires in the background. This leaves Lain with the question, how do I enter the wired? Suicide or computers. Spoilers, she picks computers. This show really succeeds with the idea that less is more. It really leaves you wondering what is real and what isn't. The show is really slow paced though and while I have no problem with that, I can see how it might bore someone else. There were at least 2 death in this episode. 
		\subsection{Layer 02}
			This episode starts with a dude dropping accella some form of techno drug which apparently causes the brain to speed up twelve times. It then changes to Lain's sister asking if Chisa Yomoda is in Lain's room. She then goes on to imply that Chisa is something like an imaginary friend. After that we have the introduction of the first agent. Lain's friends appear asking about where Lain was last night and talk about someone who just looks like Lain. This person seen seems like an excuse to get Lain into Cyberia. Lain then trips balls on the way home to get a new computer. The new computer runs the Copland OS Enterprise operating system created by the Tachibana lab. It uses 3D models in it's interface. Lain then goes to Cyberia. The episode ends with 2 people dying from the accella thing and then he commits suicide after saying "The wired can never interfere with the real world." Lain responds saying "No matter where you are, everyone is always connected" This causes accella boy to off himself. This episode has a little more aggressive feel to it. The title is girls and it really give you the feeling of competition going on between them all. It didn't seem like there was much of a point to this episode. It's more just like a list of things that happened then anything. 
\end{document}