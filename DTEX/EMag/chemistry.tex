\documentclass[12pt]{extarticle}
\usepackage{amsmath}
\usepackage{tikz}

\usepackage{amsfonts}
\usepackage{amssymb}
\usepackage{graphicx}
\usepackage{mathtools}
\usepackage{color}
\usepackage{hyperref}

\usepackage[margin=0.5in]{geometry}
\newcommand\tab[1][1cm]{\hspace*{#1}}

\usepackage{amsthm}

\theoremstyle{Definition}
\newtheorem{def.}{Definition}[section] 

\title{Chemistry}
\begin{document}
\maketitle
\begin{def.} \textbf{Matter and Substances} \\ 
	Anything that has Mass and Volume. 
	$$M = \{p : p \text{ has Mass: } p_m \text{ and has Volume: } p_V\}$$ 
	Then $M$ is the set of all matter and all $p \in M$ are said to be substances. 
\end{def.}
\begin{def.} \textbf{Chemistry} \\
	The study of Matter, it's properties, and how it changes. 
\end{def.}
\begin{def.} \textbf{Composition} \\ 
	The composition of a substance p is the type and amount of simpler substances which make up p. 
\end{def.}
\begin{def.} \textbf{States} \\ 
	A substance in a \textbf{solid state} has a fixed shape disregarding it's container. The particles in a solid will appear directly next to each other in an array. \\ \\
	A substance in a \textbf{liquid state } doesn't have a fixed shape and will fill a container but will have a fixed volume. The particles will appear randomly around each other. \\ \\
	A substance in a \textbf{gas state} doesn't have fixed shape and will fill a container will not have a fixed volume. The particles will appear randomly around each other and be separated by non-zero distances. 
\end{def.}
\begin{def.}\textbf{Properties} \\ 
	The properties of a substance are characteristics of the substance which do not changing into or interacting with another substance. \\ 
\end{def.}
\newpage
\noindent
\section{Orbitals}
Four quantum numbers can describe an electron in an atom completely:
\begin{itemize}
	\item Principal quantum number $(n)$
	\item Azimuthal quantum number $(l)$
	\item Spin quantum number $(s)$
	\item Magnetic quantum number $(ml)$
\end{itemize}
\href{https://en.wikipedia.org/wiki/Quantum_number}{Reference} \\ \\
\noindent
An important problem in quantum mechanics is that of a particle in a spherically symmetric potential, i.e., a potential that depends only on the distance between the particle and a defined center point. In particular, if the particle in question is an electron and the potential is derived from Coulomb's law, then the problem can be used to describe a hydrogen-like (one-electron) atom (or ion).
 \\ \\
In the general case, the dynamics of a particle in a spherically symmetric potential are governed by a Hamiltonian of the following form: 
$$\hat{H} = \frac{\hat{p}}{2m_0} + V(r)$$
Where $\hat{p}$ is the momentum operator, $m_0$ is the mass of the particleand the potential $V(r)$ depends only on  r, the length of the radius vector $\vec{r}$.  \\ \\
The quantum mechanical wavefunctions and energies (eigenvalues) are found by solving the Schrödinger equation with this Hamiltonian. \\ \\
The eigenstates of the system have the form
$$\psi (r,\theta ,\phi )=R(r)\Theta (\theta )\Phi (\phi )$$
in which the spherical polar angles $\theta$ and $\phi$ represent the colatitude and azimuthal angle, respectively. The last two factors of $\psi$ are often grouped together as spherical harmonics, so that the eigenfunctions take the form: 
$$\psi (r,\theta ,\phi )=R(r)Y_{lm}(\theta ,\phi )$$
The differential equation which characterizes the function $R(r)$ is called the radial equation. \\
\href{https://en.wikipedia.org/wiki/Particle_in_a_spherically_symmetric_potential}{Reference} \\ \\
In addition to l and m, a third integer $n > 0$, emerges from the boundary conditions placed on R. The functions R and Y that solve the equations above depend on the values of these integers, called quantum numbers. \\ 
$$\psi (r,\theta ,\phi )=R_{nl}(r)Y_{lm}(\theta ,\phi )$$
$$R_{nl}(r) = \sqrt{\left(\frac{2Z}{na_\mu}\right)^3\frac{(n-l-1)!}{2n(n+1)!}}e^{\frac{-Zr}{na_\mu}}\left(\frac{2Zr}{na_\mu}\right)L_{n-l-1}^{2l+1}\left(\frac{2Zr}{na_\mu}\right) $$
Where Z is the atomic number (number of protons in the nucleus), \\
e is the elementary charge (charge of an electron), \\
$L_n^\alpha$ is a Generalized Laguerre polynomial, \\
$\alpha_\mu = \frac{m_e}{\mu}a_0$ \\
$\alpha_0 = 5.29177210903\times10^{-11} m$ is the \href{https://en.wikipedia.org/wiki/Bohr_radius}{Bohr radius} \\
$\mu = \frac{m_Nm_e}{m_N + m_e} \approx m_e = 9.1093837015(28)\times10^{-31} kg$ which is the mass of an electron. \\
Where $m_N$ is the mass of a nucleus. \\ 
Where $Y_{lm}(\theta ,\phi )$ is a spherical harmonic. \\ 
\href{https://en.wikipedia.org/wiki/Hydrogen-like_atom}{Reference} \\ \\
\subsection{Generalized Laguerre polynomial}
For arbitrary real $\alpha$ the polynomial solutions of the following differential equation: 
$$xy'' + (\alpha +1 - x)y' + ny = 0$$
are called generalized Laguerre polynomials. \\
A Recursive formulation for the Generalized Laguerre polynomials is: \\
$$L_{0}^{(\alpha )}(x)=1
$$
$$L_{1}^{(\alpha )}(x)=1+\alpha -x
$$
and then using the following recurrence relation for any k $\geq$ 1:
$$L_{k+1}^{(\alpha )}(x)={\frac {(2k+1+\alpha -x)L_{k}^{(\alpha )}(x)-(k+\alpha )L_{k-1}^{(\alpha )}(x)}{k+1}}$$
\subsection{Spherical Harmonics}
\[
Y_{lm} (\theta,\phi) = 
\begin{Bmatrix}
(-1)^m\sqrt{2}\sqrt{\frac{2l+1}{4\pi}\frac{(l-|m|)!}{(l+|m|)!}}P_l^{|m|}(cos(\theta)) sin(|m|\phi) & m < 0 \\
\sqrt{\frac{2l+1}{4\pi}}P_l^{0}(cos(\theta)) & m = 0 \\
(-1)^m\sqrt{2}\sqrt{\frac{2l+1}{4\pi}\frac{(l-m)!}{(l+m)!}}P_l^{m}(cos(\theta)) cos(m\phi) & m > 0 
\end{Bmatrix}
\]
Where $P_l^m(x)$ is the Associated Legendre Polynomial. \\ 
$$P_l^m(x) = (-1)^m*2^l*(1-x^2)^{\frac{m}{2}}*\sum_{k=m}^l\frac{k!}{(k-m)!}*x^{k-m}* \binom{l}{k}\binom{l+k-1}{l}$$
\href{https://en.wikipedia.org/wiki/Spherical_harmonics}{Spherical Harmonics Reference} \\ 
\href{https://en.wikipedia.org/wiki/Associated_Legendre_polynomials}{Associated Legendre polynomials Reference} \\ 


\end{document}



