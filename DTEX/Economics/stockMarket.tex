\documentclass[12pt]{extarticle}
\usepackage{amsmath}
\usepackage{amsfonts}
\usepackage{amssymb}
\usepackage{graphicx}
\usepackage{mathtools}
\usepackage{color}
\usepackage{hyperref}

\usepackage[margin=0.5in]{geometry}
\newcommand\tab[1][1cm]{\hspace*{#1}}
\usepackage{collectbox}
\DeclarePairedDelimiter\ceil{\lceils}{\rceil}
\DeclarePairedDelimiter\floor{\lfloor}{\rfloor}
\usepackage{amsthm}
%opening
\title{Stock Markets}
\author{Daniel Drake}

\theoremstyle{plain}
\newtheorem{thm}{Theorem}[section] 

\theoremstyle{Definition}
\newtheorem{def.}{Definition}[section] 

\theoremstyle{Definition}
\newtheorem{prf}{Proof}[section] 

\theoremstyle{plain}
\newtheorem{exmp}{Example}[section]

\begin{document}
	\maketitle

	\section{Introduction}
	Before we get into strategies such as the wheel, we need to have a basic model for what stocks are and how they work. 
	
	\section{Symbols}
	$\mathbb{R} = (-\infty,\infty)$ \\ 
	$\mathbb{R}^+ = [0,\infty)$

	\section{Stock Market}
	
	Suppose that we are looking at the stock value for company E. \\
	A stock is a function $S_E : T \to V$ \\
	where T is set of possible times, $T = \mathbb{R}^+$ and V is the set of possible values. $V = \mathbb{R}^+$ \\ 

	The function $S_E$ is continuous almost everywhere meaning that most of the time, there will only be rather small changes in value, but at any point, there could be large changes in value. \\

	The time $t_0$ is the point in time when the company goes public. \\

	The time $\tau$ is the current point in time. \\
	Time 0 is the start of the universe, so of course, most stocks had zero value for most of existence. \\

	We can know the value of $S_E$ on the set $[t_0,\tau]$ \\
	
	\section{Step 1}
	Sell cash covered put. \\ \\


	
\end{document}