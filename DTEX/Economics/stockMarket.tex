\documentclass[12pt]{extarticle}
\usepackage{amsmath}
\usepackage{amsfonts}
\usepackage{amssymb}
\usepackage{graphicx}
\usepackage{mathtools}
\usepackage{color}
\usepackage{hyperref}

\usepackage[margin=0.5in]{geometry}
\newcommand\tab[1][1cm]{\hspace*{#1}}
\usepackage{collectbox}
\DeclarePairedDelimiter\ceil{\lceils}{\rceil}
\DeclarePairedDelimiter\floor{\lfloor}{\rfloor}
\usepackage{amsthm}
%opening
\title{Stock Markets}
\author{Daniel Drake}

\theoremstyle{plain}
\newtheorem{thm}{Theorem}[section] 

\theoremstyle{Definition}
\newtheorem{def.}{Definition}[section] 

\theoremstyle{Definition}
\newtheorem{prf}{Proof}[section] 

\theoremstyle{plain}
\newtheorem{exmp}{Example}[section]

\begin{document}
	\maketitle

	\section{Introduction}
	A basic model for the stock market. 
	
	\section{Symbols}
	$\mathbb{R} = (-\infty,\infty)$ \\ 
	$\mathbb{R}^+_0 = [0,\infty)$ \\
	$\in$ means "in" or "included in" \\
	$\exists$ means there exists.
	\section{Stock Market}
	Suppose that we are looking at the stock value for company E. \\ \\
	A stock is a function $S_E : T \to \mathbb{R}^+_0$ where $T \subset \mathbb{R}^+_0$ and $0 < card(T) < card(\mathbb{N})$ and where $1 \in \mathbb{R}^+_0$ is $1$ second from 0. 
	Further, there exists a time $\tau_0 \in T$ such that $\tau_0 \leq t$ for all $t \in T$. 
	The time $\tau_0$ is the point in time when the company E went public and the value $S_E(\tau_0)$ is the initial value of the companies stock. \\ \\
	
	


	
\end{document}