\documentclass[12pt]{extarticle}
\usepackage{amsmath}
\usepackage{amsfonts}
\usepackage{amssymb}
\usepackage{graphicx}
\usepackage{mathtools}
\usepackage{color}
\usepackage{hyperref}

\usepackage[margin=0.5in]{geometry}
\newcommand\tab[1][1cm]{\hspace*{#1}}
\usepackage{collectbox}
\DeclarePairedDelimiter\ceil{\lceils}{\rceil}
\DeclarePairedDelimiter\floor{\lfloor}{\rfloor}
\usepackage{amsthm}
%opening
\title{CVS CLI Documentation}
\author{Daniel Drake}

\theoremstyle{plain}
\newtheorem{thm}{Theorem}[section] 

\theoremstyle{Definition}
\newtheorem{def.}{Definition}[section] 

\theoremstyle{Definition}
\newtheorem{prf}{Proof}[section] 

\theoremstyle{plain}
\newtheorem{exmp}{Example}[section]

\begin{document}
	\maketitle
	\section{Introduction}
	cvs is a command in Linux used to do version control.  \\
	cvs has the following command syntax: 
	\begin{center}
		\verb|cvs [cvs_options] cvs_command [command_options] [command_args]|
	\end{center}

	\section{[cvs\_options]}

	\section{cvs\_command}
	The essential commands are: 
	\begin{itemize}
		\item checkout \\
		This option creates a private copy of the source. 
		\item update \\
		This option will update your source files with the changes other developers have made. 
		\item add \\
		Adds new files in the cvs records of your working directory. 
		\item remove \\
		Removes files from the repository. 
		\item commit \\
		Publishes your changes to the source repository. 
	\end{itemize}
	\section{[command\_options]}
	
	\section{[command\_args]}
	
\end{document}